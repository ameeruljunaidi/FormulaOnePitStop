% Do not change any of these lines below

\documentclass[11pt]{article}

\usepackage[T1]{fontenc}                   % Package needed for fonts
\usepackage[utf8]{inputenc}                % Package needed for fonts
\usepackage{mathpazo}                      % Use the mathpazo font
\usepackage{setspace}                      % Allows easy changes to line spacing 
\usepackage{graphicx}                      % Allows including of graphics files
\usepackage{amsmath}                       % Additional math capabilities
\usepackage{datetime}                      % Allows formatting of date and time
\usdate                                    % Use usual LaTeX date layout
\usepackage{enumitem}                      % Change formatting of lists
\usepackage{subfigure}                     % Create numbered and captioned subfigures
\usepackage{rotating}                      % Create landscape tables and figures
\usepackage{hyperref}                      % URLS and hyperlinks
\usepackage{float}                         % Activate [H] option to place figure HERE
\usepackage[longnamesfirst]{natbib}        % Bibliography formatting
\usepackage[english]{babel}                % To get blindtext
\usepackage{blindtext}                     % To generate random text
\usepackage[margin=1in]{geometry}          % Set margins
\usepackage{pythontex}                     % Load the python package
\usepackage{mdframed}                      % Can add frames around stuff with this package
\usepackage{listings}                      % To show python codes
\usepackage{booktabs}                      % To format tables
\usepackage{parskip}                       % Create line breaks on new paragraphs
\usepackage[dvipsnames]{xcolor}            % Package for color
\usepackage{verbatim}                      % To copy text verbatim to tex

\renewcommand{\baselinestretch}{1}         % Set single spacing

\usepackage{silence}                       % Disable all warnings issued by latex starting with
\WarningFilter{latex}{You have requested package}

\setlength\parindent{0pt}                  % Set no indent
\setlength{\parskip}{1em}                  % Set space between paragraphs
\graphicspath{ {./figures/} }              % Set image path
\urlstyle{same}                            % Set url setting

% Set hyperlink colors
\hypersetup{
    colorlinks=true,
    linkcolor=blue,
    filecolor=blue,      
    urlcolor=blue,
    citecolor=blue,
}

% Create custom color for mdframed
\definecolor{light-gray}{gray}{0.95}

% Packages included specifically for this document.
\usepackage{./sty/texintro}                % Document-specific definitions
\usepackage{tocvsec2}                      % More flexible formatting of table of contents
\usepackage{bibentry}                      % Print full citation in text
\usepackage[retainorgcmds]{IEEEtrantools}  % Equation formatting. Option needed to allow enumitem to work.

% These next lines allow including or excluding different versions of text using versionPO.sty
\usepackage{./sty/versionPO}               % Include text conditionally
\includeversion{links}                     % Turn hyperlinks on
\includeversion{toc}		               % Include table of contents

\let\oldmarginpar\marginpar                % Save original definition of \marginpar
\setcounter{tocdepth}{2}                   % Number paragraphs and subparagraphs and include them in TOC
      % Load all the necessary packages
\begin{pycode}
import pythoncodes
\end{pycode}
   % Load all the python codes
\begin{document}           % Begin document
\nobibliography*                            % Allow use of \bibentry command
\setlist{noitemsep}                         % Reduce space between list items 

% Define commands for title and abstract
\newcommand{\tit}{\textbf{Misbehaving In The Pits}}

\newcommand{\abs}{
\blindtext
}

\title{\tit}

\author{John Doe}

\date{\today}

\maketitle

\thispagestyle{empty}

\begin{abstract}
    \noindent
    \abs
\end{abstract}

\cleardoublepage

% Select spacing
\onehalfspacing

% Print table of contents (if desired)
\begin{toc}
\thispagestyle{empty}
\begin{singlespacing}
\tableofcontents
\end{singlespacing}
\clearpage
\end{toc}

\setcounter{page}{1}
     % Add title, abstract, toc, etc.

% Only begin writing after this line

\section{Points system and the pit stop from the strategic point of view}

\subsection{Basic Rules and Point System}

In Formula 1, there are 2 distinct championship competition that simultaneously takes
place during the season; driver's championship and the constructor's championship. All
the racers will be ranked based on the accumulation of individual points that results
throughout the season. Meanwhile, the constructor's championship title will be given to
the teams who achieves the highest accumulated points from their 2 drivers. Therefore,
often team have to persecute difficult strategic order, that doesn't coincide with
driver's interest, prioritizing the constructors rather than the drivers championship.

Through out the season, each racers will be participating in 21 races (During 2021
season, due to special COVID-19 procedure, the season ended with shorter 17 races). The
points will be distributed according to the ascending order of the finishing position
from each races, and those accumulation of those outcome will determine the result of
both constructors and driver's championship. 

\begin{table}[h]
\centering
\caption{Shows points awarded to drivers for every Grand Prix based on their position at the end of the race. Since there are 20 drivers on the grid, half of the drivers on the grid will get no points and drivers that not in the points will do whatever they can to get in the points, including taking extra pit stops to get fresher tires to catch up with the drivers right ahead of them.}
\label{tab:pos-points}
\vspace{2em}
\begin{tabular}{ccc}
\toprule
Position &  Points \\
\midrule
     1st &      25 \\
     2nd &      18 \\
     3rd &      15 \\
     4th &      12 \\
     5th &      10 \\
     6th &       8 \\
     7th &       6 \\
     8th &       4 \\
     9th &       2 \\
    10th &       1 \\
\bottomrule
\end{tabular}
\end{table}


The list below are the key factors that determine team's strategic decision during the
races.

Amongst the lists of strategic factors, decisions surrounding pit stops are considered
to be one of the crucial ones. During the races, each drivers are required to take at
least 1 pit stop to replace their tyre. There are 5 types of the tyres that differs based
on the softness of the rubber. The softer tyres will provide faster lap time, while
compromising on the durability which forces team to pit more often. Pit stops are
considered both risk and opportunity, since having fresher tyres will allow faster lap
time, but each pit stops will add about 19 seconds on the lap time. Therefore, team
managers will need to find ideal timing to utilize pit stop that minimize the longer lap
time, while not giving up position against other competitors.

\subsection{Potential for behavioural biases}

\label{sec:leader-delta}
Pit stop biases

Pit stop is one of the critical elements in the motor sports for teams to reach higher
position within the championship. By changing to newer tires, the car can take advantage
of the weather and road conditions, as well as, switch to fresh tyre that allows better
grip in the corner, thus resulting with higher chances of overtaking. Pit stops
decisions can drastically overturn the position, but simultaneously could contain large
risks of losing positions due to slow tyre swap or choosing the wrong tyre that does not
match the road conditions under extreme weather.

\begin{enumerate}
\setlength\parskip{1em}
\setlength\itemsep{1em}
    
    \item Overconfidence in double stacking
    During the 2020 season, the running pole setting team, Mercedes AMG  made critical
    errors in their tyre swap. The incident occurred during the lap 63, the rookie driver
    in Williams Racing, J.Aitken crushed into a wall which caused yellow flag. Often,
    yellow flags are considered to be perfect opportunities for team to return to the
    pit to replace their tyre with fresher ones without losing the position. When the
    yellow flag was waved, Mercedes's strategist quickly ordered both driver to
    immediately get back to the pit for tyre swap. This strategy when two drivers are
    ordered to simultaneously return to pit, is called "Double Stacking". Usually double
    stacking can cause chaos in the pit because the team strategists need to precisely
    schedule the driver's return to avoid overlap which can be detrimental to  smoothly
    pit-stop. This strategy should be employed when the team strategist and the pit crew
    have strong confidence in executing such high risk strategy. The pit stops is
    orchestrated by the symphony between pit crew, driver and the strategists' perfect
    coordination. However, in the Sakhir Grand Prix, Mercedes's decision on sudden
    double stacking caused chaos and confusion amongst unprepared crew, resulting with
    miss-placement of tyres and drop in position. 
    
    \item Multi 21 
    Another strategic decision regarding the pit stop in known as "Multi 21". Team
    strategist often orders the driver to swap position with their teammate, which
    believed to have better outcome for the constructor's points.

    How do you call biases that make people feel overly confident on taking chances when
    they are approaching the end of the game? This is the same theme as the Hockey
    team's behavioural biases. The teams often strategically use high risk strategy to
    take the chances. \citep{asness2018pulling}
    
    However, statistical finding indicates that  probability of result in better outcome
    have not been shown. 
    
    \item Different measurement of the performance 
    \begin{itemize}
        \item Position 
        \item Lap time delta against the pole position 
    \end{itemize}
    
\end{enumerate}

\subsection{Research Approach}

To examine the effectiveness of the pit stop strategy, there needs to be an 

\section{Regression Analyses}

\subsection{All Pit Stops}

\begin{table}[ht]
\begin{center}
\begin{tabular}{lclc}
\toprule
\textbf{Dep. Variable:}    &     pos\_chg     & \textbf{  R-squared:         } &     0.009  \\
\textbf{Model:}            &       OLS        & \textbf{  Adj. R-squared:    } &     0.009  \\
\textbf{Method:}           &  Least Squares   & \textbf{  F-statistic:       } &     68.83  \\
\textbf{Date:}             & Sun, 18 Apr 2021 & \textbf{  Prob (F-statistic):} &  1.27e-16  \\
\textbf{Time:}             &     05:23:25     & \textbf{  Log-Likelihood:    } &   -17790.  \\
\textbf{No. Observations:} &        7279      & \textbf{  AIC:               } & 3.558e+04  \\
\textbf{Df Residuals:}     &        7277      & \textbf{  BIC:               } & 3.560e+04  \\
\textbf{Df Model:}         &           1      & \textbf{                     } &            \\
\textbf{Covariance Type:}  &    nonrobust     & \textbf{                     } &            \\
\bottomrule
\end{tabular}
%\caption{OLS Regression Results}
\end{center}\begin{center}
\begin{tabular}{lcccccc}
\toprule
                   & \textbf{coef} & \textbf{std err} & \textbf{t} & \textbf{P$> |$t$|$} & \textbf{[0.025} & \textbf{0.975]}  \\
\midrule
\textbf{Intercept} &       0.9939  &        0.071     &    13.947  &         0.000        &        0.854    &        1.134     \\
\textbf{stop}      &      -0.2957  &        0.036     &    -8.296  &         0.000        &       -0.366    &       -0.226     \\
\bottomrule
\end{tabular}
\end{center}\begin{center}
\begin{tabular}{lclc}
\toprule
\textbf{Omnibus:}       & 963.412 & \textbf{  Durbin-Watson:     } &    2.204  \\
\textbf{Prob(Omnibus):} &   0.000 & \textbf{  Jarque-Bera (JB):  } & 9564.628  \\
\textbf{Skew:}          &   0.275 & \textbf{  Prob(JB):          } &     0.00  \\
\textbf{Kurtosis:}      &   8.589 & \textbf{  Cond. No.          } &     5.26  \\
\bottomrule
\end{tabular}
\end{center}
\caption{OLS Regression for all pit stops.}
\label{tab:all_pits_reg}
\end{table}

\subsection{"Normal" Pit Stops}

\begin{table}[ht]
\begin{center}
\begin{tabular}{lclc}
\toprule
\textbf{Dep. Variable:}    &     pos\_chg     & \textbf{  R-squared:         } &     0.011  \\
\textbf{Model:}            &       OLS        & \textbf{  Adj. R-squared:    } &     0.011  \\
\textbf{Method:}           &  Least Squares   & \textbf{  F-statistic:       } &     62.79  \\
\textbf{Date:}             & Sun, 18 Apr 2021 & \textbf{  Prob (F-statistic):} &  2.76e-15  \\
\textbf{Time:}             &     05:23:25     & \textbf{  Log-Likelihood:    } &   -14212.  \\
\textbf{No. Observations:} &        5625      & \textbf{  AIC:               } & 2.843e+04  \\
\textbf{Df Residuals:}     &        5623      & \textbf{  BIC:               } & 2.844e+04  \\
\textbf{Df Model:}         &           1      & \textbf{                     } &            \\
\textbf{Covariance Type:}  &    nonrobust     & \textbf{                     } &            \\
\bottomrule
\end{tabular}
%\caption{OLS Regression Results}
\end{center}\begin{center}
\begin{tabular}{lcccccc}
\toprule
                   & \textbf{coef} & \textbf{std err} & \textbf{t} & \textbf{P$> |$t$|$} & \textbf{[0.025} & \textbf{0.975]}  \\
\midrule
\textbf{Intercept} &       1.3037  &        0.097     &    13.465  &         0.000        &        1.114    &        1.494     \\
\textbf{stop}      &      -0.4738  &        0.060     &    -7.924  &         0.000        &       -0.591    &       -0.357     \\
\bottomrule
\end{tabular}
\end{center}\begin{center}
\begin{tabular}{lclc}
\toprule
\textbf{Omnibus:}       & 638.114 & \textbf{  Durbin-Watson:     } &    2.171  \\
\textbf{Prob(Omnibus):} &   0.000 & \textbf{  Jarque-Bera (JB):  } & 5161.235  \\
\textbf{Skew:}          &   0.227 & \textbf{  Prob(JB):          } &     0.00  \\
\textbf{Kurtosis:}      &   7.671 & \textbf{  Cond. No.          } &     5.17  \\
\bottomrule
\end{tabular}
\end{center}
\caption{OLS Regression for control pit stops.}
\label{tab:control_pits_reg}
\end{table}

\subsection{More-Than-Average Pit Stops}

\begin{table}[ht]
\begin{center}
\begin{tabular}{lclc}
\toprule
\textbf{Dep. Variable:}    &     pos\_chg     & \textbf{  R-squared:         } &    0.007  \\
\textbf{Model:}            &       OLS        & \textbf{  Adj. R-squared:    } &    0.007  \\
\textbf{Method:}           &  Least Squares   & \textbf{  F-statistic:       } &    11.41  \\
\textbf{Date:}             & Sun, 18 Apr 2021 & \textbf{  Prob (F-statistic):} & 0.000748  \\
\textbf{Time:}             &     05:23:25     & \textbf{  Log-Likelihood:    } &  -3505.9  \\
\textbf{No. Observations:} &        1539      & \textbf{  AIC:               } &    7016.  \\
\textbf{Df Residuals:}     &        1537      & \textbf{  BIC:               } &    7026.  \\
\textbf{Df Model:}         &           1      & \textbf{                     } &           \\
\textbf{Covariance Type:}  &    nonrobust     & \textbf{                     } &           \\
\bottomrule
\end{tabular}
%\caption{OLS Regression Results}
\end{center}\begin{center}
\begin{tabular}{lcccccc}
\toprule
                   & \textbf{coef} & \textbf{std err} & \textbf{t} & \textbf{P$> |$t$|$} & \textbf{[0.025} & \textbf{0.975]}  \\
\midrule
\textbf{Intercept} &       0.9418  &        0.201     &     4.675  &         0.000        &        0.547    &        1.337     \\
\textbf{stop}      &      -0.2292  &        0.068     &    -3.378  &         0.001        &       -0.362    &       -0.096     \\
\bottomrule
\end{tabular}
\end{center}\begin{center}
\begin{tabular}{lclc}
\toprule
\textbf{Omnibus:}       & 158.776 & \textbf{  Durbin-Watson:     } &     1.915  \\
\textbf{Prob(Omnibus):} &   0.000 & \textbf{  Jarque-Bera (JB):  } &  1100.240  \\
\textbf{Skew:}          &  -0.172 & \textbf{  Prob(JB):          } & 1.22e-239  \\
\textbf{Kurtosis:}      &   7.128 & \textbf{  Cond. No.          } &      11.0  \\
\bottomrule
\end{tabular}
\end{center}
\caption{OLS Regression for more-than-average stops.}
\label{tab:extra_pits_reg}
\end{table}

\begin{pycode}
pythoncodes.view_rank_pos()
\end{pycode}

\begin{pycode}
print("What's up")
\end{pycode}

% Do not change anything after this line
\clearpage
\bibliographystyle{./bst/jf}
\bibliography{misc/references}
\addcontentsline{toc}{section}{References}

\end{document}
